\subsection{Interpolation}
This section decribes all the routines in the file "ptwXY\_interpolation.c".

\subsubsection{ptwXY\_interpolatePoint}
\setargumentNameLengths{interpolation}
This routine interpolates an $x$ value between the points (x1,y1) and (x2,y2) to obtain its $y$ value.
\CallingC{fnu\_status ptwXY\_interpolatePoint(}{ptwXY\_interpolation interpolation,
    \addArgument{double x,}
    \addArgument{double *y,}
    \addArgument{double x1,}
    \addArgument{double y1,}
    \addArgument{double x2,}
    \addArgument{double y2 );}}
    \argumentBox{interpolation}{Type of interpolation to perform (see Section~\ref{interpolationSection}).}
    \argumentBox{x}{The $x$ value at which the $y$ value is desired.}
    \argumentBox{x1}{The $x$ value of the first point.}
    \argumentBox{y1}{The $y$ value of the first point.}
    \argumentBox{x2}{The $x$ value of the second point.}
    \argumentBox{y2}{The $y$ value of the second point.}
    \vskip 0.05 in \noindent
If the interpolation flag is invalid or ( x1 $>$ x2 ) then \highlight{nfu\_invalid\-Interpolation} is returned. 
If logarithm interpolation is requested for an axis, and one of the input values for that axis is less than or equal to 0., 
then \highlight{nfu\_invalid\-Interpolation} is also returned. If interpolation is \highlight{ptwXY\_interpolationOther} then
\highlight{nfu\_otherInterpolation} is returned.

\subsubsection{ptwXY\_flatInterpolationToLinear}
This routine returns a linear-linear interpolated representation of \highlight{ptwXY}.
\setargumentNameLengths{upperEps}
\CallingC{ptwXYPoints *ptwXY\_flatInterpolationToLinear(}{ptwXYPoints *ptwXY,
    \addArgument{double lowerEps,}
    \addArgument{double upperEps,}
    \addArgument{nfu\_status *status );}}
    \argumentBox{ptwXY}{A pointer to a \highlight{ptwXYPoints} object.}
    \argumentBox{lowerEps}{The amount to adjust every interior point down in x.}
    \argumentBox{upperEps}{The amount to adjust every interior point up in x}
    \argumentBox{status}{On return, the status value.}
    \vskip 0.05 in \noindent
For every interior point (i.e., $x_i,y_i$ for $0 < i < n - 1$ where n is the number of points), two points are added.
The positions of these points depend on \highlight{lowerEps} and \highlight{upperEps} as follows:
\begin{description}
    \item[lowerEps ==  0 and upperEps ==  0:] This condition is not allowed. status is set to \highlight{nfu\_bad\-Input} and NULL is returned.
        This condition is also returned if either \highlight{lowerEps} or \highlight{upperEps} is negative.
    \item[lowerEps $>$ 0 and upperEps ==  0:] At each interior point $x_i,y_i$ the two points $x_m,y_{i-1}$ and $x_i,y_i$ are set.
    \item[lowerEps ==  0 and upperEps $>$ 0:] At each interior point $x_i,y_i$ the two points $x_i,y_{i-1}$ and $x_p,y_i$ are set.
    \item[lowerEps $>$ 0 and upperEps $>$ 0:] At each interior point $x_i,y_i$, this point is removed and the two 
        points $x_m,y_{i-1}$ and $x_p,y_i$ are set.
\end{description}
where $x_m$ and $x_p$ are given in Table~\ref{flatInterpolationToLinear}.
\begin{table}
\begin{center}
\begin{tabular}{|c|l|l|}  \hline
                & $x_m$                    & $x_p$                          \\ \hline \hline
    $x_i <  0$  & $x_i ( 1 + \epsilon_l )$ & $x_p = x_i ( 1 - \epsilon_p )$ \\ \hline
    $x_i == 0$  & $ -\epsilon_l $          & $ \epsilon_p $                 \\ \hline
    $x_i >  0$  & $x_i ( 1 - \epsilon_l )$ & $x_p = x_i ( 1 + \epsilon_p )$ \\ \hline
\end{tabular}
\end{center}
\caption{The value of $x_m$ and $x_p$ used to adjust interior points in \highlight{ptwXY\_fla-Interpolation\-To\-Linear}. 
    Here, $ \epsilon_l = $ \highlight{lowerEps} and $ \epsilon_p = $ \highlight{upperEps}. \label{flatInterpolationToLinear}}
\end{table}

\subsubsection{ptwXY\_toOtherInterpolation}
This routine returns \highlight{ptwXY} converted to interpolation \highlight{interpolation}.
\setargumentNameLengths{interpolation}
\CallingC{ptwXYPoints *ptwXY\_toOtherInterpolation(}{ptwXYPoints *ptwXY,
    \addArgument{ptwXY\_interpolation interpolation,}
    \addArgument{double accuracy,}
    \addArgument{nfu\_status *status );}}
    \argumentBox{ptwXY}{A pointer to a \highlight{ptwXYPoints} object.}
    \argumentBox{interpolation}{The interpolation to convert to.}
    \argumentBox{accuracy}{The accuracy of the conversion.}
    \argumentBox{status}{On return, the status value.}
    \vskip 0.05 in \noindent
Currently, \highlight{interpolation} can only be \highlight{ptwXY\_\-interpolation\-LinLin}.

\subsubsection{ptwXY\_toUnitbase}
This routine returns a unit-based version of \highlight{ptwXY}.
\setargumentNameLengths{status}
\CallingC{ptwXYPoints *ptwXY\_toUnitbase(}{ptwXYPoints *ptwXY,
    \addArgument{nfu\_status *status );}}
    \argumentBox{ptwXY}{A pointer to the \highlight{ptwXYPoints} object.}
    \argumentBox{status}{On return, the status value.}
    \vskip 0.05 in \noindent
Unitbasing maps the domain to 0 to 1 by scaling each x-value as $x_i = ( x_i - x_0 ) / ( x_{n-1} - x_0 )$ and scaling each y-value as
$y_i = y_i \times ( x_{n-1} - x_0 )$. Unitbasing is most useful on pdf's.

\subsubsection{ptwXY\_fromUnitbase}
This routine undoes the unit base mapping done by \highlight{ptwXY\_toUnitbase}.
\setargumentNameLengths{accuracy}
\CallingC{ptwXYPoints *ptwXY\_fromUnitbase(}{ptwXYPoints *ptwXY,
    \addArgument{double xMin,}
    \addArgument{double xMax,}
    \addArgument{nfu\_status *status );}}
    \argumentBox{ptwXY}{A pointer to the \highlight{ptwXYPoints} object.}
    \argumentBox{xMin}{The lower domain for the returned \highlight{ptwXYPoints} instances.}
    \argumentBox{xMax}{The upper domain for the returned \highlight{ptwXYPoints} instances.}
    \argumentBox{status}{On return, the status value.}
    \vskip 0.05 in \noindent
Each x-value is scaled as $x_i = ( {\rm xMax} - {\rm xMax} ) \times x_i + {\rm xMax}$ and each y-value is scaled 
as $y_i = y_i / ( {\rm xMax} - {\rm xMax} )$.

\subsubsection{ptwXY\_unitbaseInterpolate}
This routine returns a \highlight{ptwXYPoints} instance that is the unit-base interpolation of \highlight{ptwXY1} at $w_1$
and \highlight{ptwXY2} at $w_2$ at the w-value $w$.
\setargumentNameLengths{interpolation}
\CallingC{nfu\_status ptwXY\_unitbaseInterpolate(}{double w,
    \addArgument{double w1,}
    \addArgument{ptwXYPoints *ptwXY1,}
    \addArgument{double w2,}
    \addArgument{ptwXYPoints *ptwXY2,}
    \addArgument{nfu\_status *status );}}
    \argumentBox{w}{The w-value to interpole to.}
    \argumentBox{w1}{The lower w-value}
    \argumentBox{ptwXY1}{A pointer to a \highlight{ptwXYPoints} object at w1.}
    \argumentBox{w2}{The upper w-value}
    \argumentBox{ptwXY2}{A pointer to a \highlight{ptwXYPoints} object at w2.}
    \argumentBox{status}{On return, the status value.}
    \vskip 0.05 in \noindent
