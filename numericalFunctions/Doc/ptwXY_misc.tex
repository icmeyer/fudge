\subsection{Miscellaneous}
This section decribes all the routines in the file "ptwXY\_misc.c".

\subsubsection{ptwXY\_update\_biSectionMax --- Not for general use}
This routine is used by \highlight{ptwXY} routines to update the member \highlight{biSectionMax} base on the prior length, given
by \highlight{oldLength}, and the current length of \highlight{ptwXY}.
\CallingCLimited{void ptwXY\_update\_biSectionMax(}{ptwXYPoints *ptwXY,
    \addArgument{double oldLength );}}
    \argumentBox{ptwXY}{A pointer to the \highlight{ptwXYPoints} object.}
    \argumentBox{oldLength}{.}

\subsubsection{ptwXY\_applyFunction}
This routine is used by other routines to map $y_i$ to func( $x_i$, $y_i$ ) with infilling as needed. For example,
this routine is used by \highlight{ptwXY\_pow}.
\CallingC{nf\_status ptwXY\_applyFunction(}{ptwXYPoints *ptwXY,
    \addArgument{ptwXY\_applyFunction\_callback func,}
    \addArgument{void *argList );}}
    \argumentBox{ptwXY}{A pointer to the \highlight{ptwXYPoints} object.}
    \argumentBox{func}{A function called to calculate func( $x_i$, $y_i$ ).}
    \argumentBox{argList}{A pointer passed to \highlight{func}.}
This routine infills to maintain the initial accuracy. The function \highlight{func} called as if defined as 
\begin{verbatim}
    nfu_status func( ptwXYPoint *point, void *argList );    .
\end{verbatim}

\subsubsection{ptwXY\_showInteralStructure --- Not for general use}
This routine writes out details of the data in a \highlight{ptwXYPoints} object, including much of the
internal data normally not useful to a user. This routine is intended for debugging.
\setargumentNameLengths{printPointersAsNull}
\CallingCLimited{void ptwXY\_showInteralStructure(}{ptwXYPoints *ptwXY,
    \addArgument{FILE *f,}
    \addArgument{int printPointersAsNull );}}
    \argumentBox{ptwXY}{A pointer to the \highlight{ptwXYPoints} object.}
    \argumentBox{f}{The stream where the structure is written.}
    \argumentBox{printPointersAsNull}{If true, all pointers are printed as if their value is NULL.}

\subsubsection{ptwXY\_simpleWrite}
This routine writes out the (x,y) points of the \highlight{ptwXYPoints} object to a specified stream.
\setargumentNameLengths{format}
\CallingC{void ptwXY\_simpleWrite(}{ptwXYPoints *ptwXY,
    \addArgument{FILE *f,}
    \addArgument{char *format );}}
    \argumentBox{ptwXY}{A pointer to the \highlight{ptwXYPoints} object.}
    \argumentBox{f}{The stream where the points are written.}
    \argumentBox{format}{The format specifier to use for writing an (x,y) point.}
    \vskip 0.05 in \noindent
The \highlight{format} must contain two C double specifier (e.g., "\%12.4f \%17.7e$\backslash$n"), one each for the x- and y-values of a point.
No line feed characters (e.g., "$\backslash$n") are printed, except those in \highlight{format}.

\subsubsection{ptwXY\_simplePrint}
This routine calls \highlight{ptwXY\_simpleWrite} with stdout as the output stream.
\CallingC{void ptwXY\_simplePrint(}{ptwXYPoints *ptwXY,
    \addArgument{char *format );}}
    \argumentBox{ptwXY}{A pointer to the \highlight{ptwXYPoints} object.}
    \argumentBox{format}{The format specifier to use for writing an (x,y) point.}
